\begin{table}[h!]
\begin{tabular}{| p{0.18\linewidth} | p{0.72\linewidth} |}
\hline
\textbf{ID} & TF001v1 \\ \hline
\textbf{Namn} & Test av dörrlarm\\ \hline
\textbf{Beskrivning} & Syfte: Verifiera att dörrlarmet fungerar enligt spefikation i projektrapporten. \newline
Komponent: P1 (steg 1-3) och C (steg 4-5) \newline
Krav: K01 (steg 1-3) och K02 (steg 4-5) \\ \hline
\textbf{Teststeg} & 1. Låt dörren stå öppen. \newline
2. Vänta tills enheten larmar lokalt. \newline
3. Vänta tills enheten larmar centralenheten. \newline
4. Mata in fel kod på centralenhetens knappsats. \newline
5. Mata in rätt kod på centralenhetens knappsats. \newline
6. Upprepa setg 1-5 flera gånger. \\ \hline
\textbf{Förväntat resultat} & Efter en viss tid ska enheten larma lokalt, dvs. en röd lysdiod ska tändas. 
Efter ytterligare en viss tid ska enheten larma centralenheten.
När fel kod matas in ska larmet fortsätta.
När rätt kod matas in ska larmet sluta.
\\ \hline
\end{tabular}
\end{table}

\begin{table}[h!]
\begin{tabular}{| p{0.18\linewidth} | p{0.72\linewidth} |}
\hline
\textbf{ID} & TF002v1 \\ \hline
\textbf{Namn} & Inaktivering och aktiviering av dörrlarm \\ \hline
\textbf{Beskrivning} & Syfte: Verifiera att inaktivering och aktivering av dörrlarm fungerar. \newline
Komponent: P1 \newline
Krav: K06 \\ \hline
\textbf{Teststeg} & 1. Inaktivera larmet på en dörr. \newline
2. Ställ upp dörren. \newline
3. Vänta.\newline
4. Stäng dörren.\newline
5. Aktivera dörrlarmet.\newline
6. Ställ upp dörren.\newline
7. Vänta tills larmet går.\newline
8. Stäng av larmet.\newline
9. Upprepa steg 1-8 flera gånger.\\ \hline
\textbf{Förväntat resultat} & När larmet är inaktiverat ska en grön lysdiod lysa på centralenheten och larmet
INTE gå när dörren står öppen.
När larmet är aktiverat ska larmet gå när dörren står öppen.
\\ \hline
\end{tabular}
\end{table}

\newpage

\begin{table}[h!]
\begin{tabular}{| p{0.18\linewidth} | p{0.72\linewidth} |}
\hline
\textbf{ID} & TF003v1 \\ \hline
\textbf{Namn} & Konfigurering av tid en dörr tillåts vara öppen \\ \hline
\textbf{Beskrivning} &
Syfte: Verifiera att tiden en dörr tillåts vara öppen kan ändras via 
centralenheten.\newline
Komponent: P1 och C\newline
Krav: K06
\\ \hline
\textbf{Teststeg} &
1. Konfigurera tiden en dörr tillåts vara öppen via centralenheten. \newline
2. Öppna dörren. \newline
3. Vänta tills larmet går. \newline
4. Upprepa steg 1-3 flera gånger med olika tider.
\\ \hline
\textbf{Förväntat resultat} & Efter att tiden som konfigurerats via centralenheten passerat ska larmet gå.
\\ \hline
\end{tabular}
\end{table}

\begin{table}[h!]
\begin{tabular}{| p{0.18\linewidth} | p{0.72\linewidth} |}
\hline
\textbf{ID} & TF004v1 \\ \hline
\textbf{Namn} & Test av avståndsmätare, rörelselarm \\ \hline
\textbf{Beskrivning} &
Syfte: Verifiera att avståndsmätaren i rörelselarmet fungerar enligt
specifikationen i projektrapporten. \newline
Komponent: P2\newline
Krav: K07
\\ \hline
\textbf{Teststeg} &
1. Kalibrera avståndsmätaren.\newline
2. Ställ in känslighet.\newline
3. Gå inom känslighetsområdet för avståndsmätaren.\newline
3. Upprepa steg 2-3 med olika känslighet.
\\ \hline
\textbf{Förväntat resultat} & Larmet ska gå när någon rör sig inom avståndsmätarens känslighetsområde.
\\ \hline
\end{tabular}
\end{table}

\begin{table}[h!]
\begin{tabular}{| p{0.18\linewidth} | p{0.72\linewidth} |}
\hline
\textbf{ID} & TF005v1 \\ \hline
\textbf{Namn} & Test av vibrationssensor, rörelselarm \\ \hline
\textbf{Beskrivning} &
Syfte: Verifiera att vibrationssensorn i rörelselarmet fungerar enligt
specifikationen i projektrapporten.\newline
Komponent: P2\newline
Krav: K09
\\ \hline
\textbf{Teststeg} &
1. Ställ in känsligheten för vibrationssensorn.\newline
2. Simulera att en glasruta krossas.\newline
3. Upprepa steg 1-2 med olika känslighet för vibrationssensorn.
\\ \hline
\textbf{Förväntat resultat} & När simuleringen av att en glasruta krossas sker, ska larmet gå.
\\ \hline
\end{tabular}
\end{table}

\newpage

\begin{table}[h!]
\begin{tabular}{| p{0.18\linewidth} | p{0.72\linewidth} |}
\hline
\textbf{ID} & TF006v1 \\ \hline
\textbf{Namn} & Test av nätverket\\ \hline
\textbf{Beskrivning} &
Syfte: Testa nätverkets funktion vid olika belastningsgrader.\newline
Komponent: Nätverk\newline
Krav: Saknas
\\ \hline
\textbf{Teststeg} &1. Anslut en störenhet till nätverket.\newline
2. Justera datavolymen störenheten ska skicka.\newline
3. Starta störenheten.\newline
4. Upprepa steg 2-3 med olika datavolym.
\\ \hline
\textbf{Förväntat resultat} & Nätverket ska klara en hög databelastning.
\\ \hline
\end{tabular}
\end{table}

\begin{table}[h!]
\begin{tabular}{| p{0.18\linewidth} | p{0.72\linewidth} |}
\hline
\textbf{ID} & TF007v1 \\ \hline
\textbf{Namn} & Test av centralenheten, uppstart\\ \hline
\textbf{Beskrivning} &
Syfte: Verifiera att centralenheten känner till anslutna enheter och deras konfigurering vid uppstart. \newline
Komponent: C\newline
Krav: K12
\\ \hline
\textbf{Teststeg} &
1. Verifiera att centralenheten känner till anslutna enheter och deras 
konfiguration vid uppstart.
\\ \hline
\textbf{Förväntat resultat} & Se teststeg 1.
\\ \hline
\end{tabular}
\end{table}

\begin{table}[h!]
\begin{tabular}{| p{0.18\linewidth} | p{0.72\linewidth} |}
\hline
\textbf{ID} & TF009v1 \\ \hline
\textbf{Namn} & Test av centralenheten, periferienhet kopplas från nätverket\\ \hline
\textbf{Beskrivning} &
Syfte: Verifiera att centralenheten larmar om en periferienhet kopplas från nätverket.\newline
Komponent: C\newline
Krav: K13
\\ \hline
\textbf{Teststeg} &
1. Anslut en periferienhet till nätverket.\newline
2. Koppla bort periferienheten från nätverket genom att dra ut 
strömförsörjningsadaptern.
\\ \hline
\textbf{Förväntat resultat} & Centralenheten ska larma.
\\ \hline
\end{tabular}
\end{table}

\newpage

\begin{table}[h!]
\begin{tabular}{| p{0.18\linewidth} | p{0.72\linewidth} |}
\hline
\textbf{ID} & TF011v1 \\ \hline
\textbf{Namn} & Test av centralenheten larmfunktion (1/2)
\\ \hline
\textbf{Beskrivning} &
Syfte: Verifiera att när ett larm uppstår kommunicerar centralenheten det till en ansluten PC via USART.\newline
Komponent: C och P1\newline
Krav: K11
\\ \hline
\textbf{Teststeg} &
1. Starta ett larm från P1.\newline
2. Upprepa steg 1 med samtliga olika sätt att starta larm på.
\\ \hline
\textbf{Förväntat resultat} & När larmet går kommunicerar centralenheten detta till en ansluten PC.
\\ \hline
\end{tabular}
\end{table}

\begin{table}[h!]
\begin{tabular}{| p{0.18\linewidth} | p{0.72\linewidth} |}
\hline
\textbf{ID} & TF012v1 \\ \hline
\textbf{Namn} & Test av centralenheten larmfunktion (2/2)
\\ \hline
\textbf{Beskrivning} &
Syfte: Verifiera att när ett larm uppstår kommunicerar centralenheten det till en ansluten PC via USART. \newline
Komponent: C och P2\newline
Krav: K11
\\ \hline
\textbf{Teststeg} &
1. Starta ett larm från P2.\newline
2. Upprepa steg 1 med samtliga olika sätt att starta larm på.
\\ \hline
\textbf{Förväntat resultat} & När larmet går kommunicerar centralenheten detta till en ansluten PC.
\\ \hline
\end{tabular}
\end{table}

\begin{table}[h!]
\begin{tabular}{| p{0.18\linewidth} | p{0.72\linewidth} |}
\hline
\textbf{ID} & TF013v1 \\ \hline
\textbf{Namn} & Test av centralenheten larmfunktion (2/2)
\\ \hline
\textbf{Beskrivning} &
Syfte: Verifiera att flera dörrar kan larmas med samma enhet.\newline
Komponent: P1\newline
Krav: K03
\\ \hline
\textbf{Teststeg} &
1. Larma flera dörrar från P1.\newline
2. Upprepa steg 1 med olika antal dörrar.
\\ \hline
\textbf{Förväntat resultat} & Flera dörrar ska kunna larmas med samma enhet.
\\ \hline
\end{tabular}
\end{table}

\begin{table}[h!]
\begin{tabular}{| p{0.18\linewidth} | p{0.72\linewidth} |}
\hline
\textbf{ID} & TF014v1 \\ \hline
\textbf{Namn} & 
Test av flera dörrenheter på nätverket.
\\ \hline
\textbf{Beskrivning} &
Syfte: Verifiera att flera dörrenheter kan kopplas upp på nätverket.\newline
Komponent: N, P1\newline
Krav: K04
\\ \hline
\textbf{Teststeg} &
1. Koppla upp flera dörrlarmsenheter på nätverket.\newline
2. Verifiera att samtliga dörrlarmsenheter fungerar.
\\ \hline
\textbf{Förväntat resultat} & Samtliga dörrlarmsenheter ska fungera.
\\ \hline
\end{tabular}
\end{table}

\newpage

\begin{table}[h!]
\begin{tabular}{| p{0.18\linewidth} | p{0.72\linewidth} |}
\hline
\textbf{ID} & TF015v1 \\ \hline
\textbf{Namn} & 
Test av konfiguration av antalet dörrar på P1.
\\ \hline
\textbf{Beskrivning} &
Syfte: Verifiera att antalet dörrar för en dörrlarmsenhet kan konfigureras.\newline
Komponent: P1\newline
Krav: K05
\\ \hline
\textbf{Teststeg} &
1. Konfigurera antalet dörrar för en dörrlarmsenhet vid uppstart.
\\ \hline
\textbf{Förväntat resultat} & Antalet dörrar ska kunna konfigureras.
\\ \hline
\end{tabular}
\end{table}

\begin{table}[h!]
\begin{tabular}{| p{0.18\linewidth} | p{0.72\linewidth} |}
\hline
\textbf{ID} & TF016v1 \\ \hline
\textbf{Namn} & 
Test av kalibrering och känslighetsjustering av rörelsemätaren.
\\ \hline
\textbf{Beskrivning} &
Syfte: Verifiera känsligheten för rörelsemätaren kan ställas in samt att den kan kalibreras.\newline
Komponent: P2 och C\newline
Krav: K08
\\ \hline
\textbf{Teststeg} &
1. Kalibrera rörelsemätaren från centralenheten.\newline
2. Ställ in känsligheten för rörelsemätaren från centralenheten.\newline
3. Verifiera att rörelsemätaren är kalibrerad och har rätt känslighet.
\\ \hline
\textbf{Förväntat resultat} & Rörelsemätaren är kalibrerad och har rätt känslighet.
\\ \hline
\end{tabular}
\end{table}

\begin{table}[h!]
\begin{tabular}{| p{0.18\linewidth} | p{0.72\linewidth} |}
\hline
\textbf{ID} & TF017v1 \\ \hline
\textbf{Namn} & 
Test av störenheten
\\ \hline
\textbf{Beskrivning} &
Syfte: Verifiera att volymen data som störenheten skickar kan justeras.\newline
Komponent: S\newline
Krav: K10
\\ \hline
\textbf{Teststeg} &
1. Ändra volymen data som störenheten kan skicka.\newline
2. Verifiera att volymen data som störenheten skickar är ändrad.
\\ \hline
\textbf{Förväntat resultat} & Volymen data som störenheten skickar är ändrad enligt specifikation.
\\ \hline
\end{tabular}
\end{table}

\newpage
\begin{table}[h!]
\begin{tabular}{| p{0.18\linewidth} | p{0.72\linewidth} |}
\hline
\textbf{ID} & TF018v1 \\ \hline
\textbf{Namn} & 
Test av centralenhetens larmfunktion
\\ \hline
\textbf{Beskrivning} &
Syfte: Verifiera att centralenhetens larmar tills en fyrsiffrig kod knappas in på knappsatsen.\newline
Komponent: C\newline
Krav: K02
\\ \hline
\textbf{Teststeg} &
1. Starta larm.\newline
2. Vänta tills centralenheten larmar.\newline
3. Mata in fel fyrsiffrig kod.\newline
4. Mata in rätt fyrsiffrig kod.\newline
5. Upprepa steg 1-4 med olika larm.
\\ \hline
\textbf{Förväntat resultat} & Larmet ska fortsätta tills rätt kod matas in.
\\ \hline
\end{tabular}
\end{table}

\begin{table}[h!]
\begin{tabular}{| p{0.18\linewidth} | p{0.72\linewidth} |}
\hline
\textbf{ID} & TP001v1 \\ \hline
\textbf{Namn} & 
Rörelsesensorns räckvidd
\\ \hline
\textbf{Beskrivning} &
Syfte: Undersök räckvidden för rörelsesensorn.\newline
Komponent: P2\newline
Krav: Saknas
\\ \hline
\textbf{Teststeg} &
1. Undersök räckvidden för rörelsesensorn.
\\ \hline
\textbf{Förväntat resultat} & Se teststeg 1.
\\ \hline
\end{tabular}
\end{table}

\begin{table}[h!]
\begin{tabular}{| p{0.18\linewidth} | p{0.72\linewidth} |}
\hline
\textbf{ID} & TP001v1 \\ \hline
\textbf{Namn} & 
Störning på CAN-bussen
\\ \hline
\textbf{Beskrivning} &
Syfte: Undersöka hur mycket störning systemet klarar på CAN-bussen.\newline
Komponent: CAN-buss\newline
Krav: Saknas
\\ \hline
\textbf{Teststeg} &
1. Undersök hur mycket störning systemet klarar på CAN-bussen.
\\ \hline
\textbf{Förväntat resultat} & Se teststeg 1.
\\ \hline
\end{tabular}
\end{table}

\begin{table}[h!]
\begin{tabular}{| p{0.18\linewidth} | p{0.72\linewidth} |}
\hline
\textbf{ID} & TA001v1 \\ \hline
\textbf{Namn} & 
Test av systemet utifrån ett användarperspektiv
\\ \hline
\textbf{Beskrivning} &
Syfte: Verifiera att systemet har en god användarupplevelse.\newline
Komponent: P1, P2, C\newline
Krav: Saknas.
\\ \hline
\textbf{Teststeg} &
1. Starta upp systemet.\newline
2. Använd systemet.
\\ \hline
\textbf{Förväntat resultat} & Systemet ska ha en god användarupplevelse.
Kommentarer ska antecknas.
\\ \hline
\end{tabular}
\end{table}

\newpage
