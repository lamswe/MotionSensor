I detta appendix sammanställs det senaste testresultatet för samtliga körda tester för projektet. \textit{ID} är testfallets ID (se appendix A), \textit{Status} är anting godkänt (G) eller underkänt (U) och \textit{Kommentar} är en fritextkommentar som testaren kan ange.

\begin{table}[h!]
\begin{tabular}{| l | l | p{0.68\linewidth} |}
\hline
\textbf{ID} & \textbf{Status} & \textbf{Kommentar} \\ \hline
TF001v1 & G & \\ \hline
TF002v1 & G & En bargraph används istället för lysdioder för releasen. \\ \hline
TF003v1 & G & \\ \hline
TF004v1 & G & \\ \hline
TF005v1 & G & \\ \hline
TF006v1 & G & \\ \hline
TF007v1 & G & \\ \hline
TF009v1 & G & \\ \hline
TF010v1 & G & \\ \hline
TF011v1 & G & \\ \hline
TF012v1 & G & \\ \hline
TF013v1 & G & \\ \hline
TF014v1 & G & \\ \hline
TF015v1 & G & \\ \hline
TF016v1 & G & \\ \hline
TF017v1 & G & \\ \hline
TF018v1 & G & \\ \hline
TP001v1 & G & Ultraljudsmätaren har en räckvidd på ca 3 meter. \\ \hline
TP002v1 & G & Centralenheten klarar ca 50 meddelanden per sekund innan terminalfönstret hänger sig. Centralenheten kan inte larma av andra enheter medan den mottager meddelanden från störenheten. \\ \hline
TA001v1 & G & Alla enheter bör skriva ut enhets-info vid uppstart. \\ \hline
\end{tabular}
\end{table}
