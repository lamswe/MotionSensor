\documentclass[a4paper]{article}

\usepackage[swedish]{babel}
\usepackage[T1]{fontenc}
\usepackage[utf8]{inputenc}

\begin{document}


\section{Bakgrund utkast}

Efterfrågan för inbrottslarm i Sverige ökade mellan 2016 och 2017 med 8\%\texttt{[1]}. Detta tyder på att det finns ett växande behov av produkter inom larm- och säkerhetsbranschen.

Ett vanligt larmsystem består av en centralstation där inställningar, larmstatus och återställning av larmet hanteras samt ett antal periferienheter som dörrlarm, fönsterlarm och rörelsedetektorer. Periferienheterna skickar kontinuerligt information till centralstationen som i sin tur tolkar informationen och uppdaterar larmstatusen.

\begin{description}
\item [Dörrlarm] Dörrlarm består vanligtvis av en sensor och en magnet\texttt{[2]} som håller kontakten för att larmet inte ska aktiveras. Dessa har kontakt hela tiden när dörren är stängd, men så fort sensorn tappar kontakt med magneten, till exempel om dörren öppnas, skickas en signal till centralstationen och larmet aktiveras.
\item [Fönsterlarm] Fönsterlarm består av vibrationssensorer. Om vibrationerna för sensorn når över en bestämd tröskel, till exempel om ett fönster krossas, skickar vibrationssensorn signaler till centralenheten som aktiverar ett larm.
\item [Rörelsedetektor] Rörelsedetektorer använder sig av ultraljud. Tiden det vid en viss tidpunkt tar för ljudvågorna att studsa tillbaka till sensorn beräknas. Denna jämförs med tiden det tagit för ljudvågorna att vid en tidigare tidpunkt studsa tillbaka till sensorn. Rörelsedetektorn kan då bedömma om något har rört sig inom dess övervakningsområde.
\end{description}

\section{Referenser}
\begin{thebibliography}{9}
\bibitem{[1]}
\textit{Säkerhetsbranschen växer i Sverige} \\https://sverigesradio.se/sida/artikel.aspx?programid=83\&artikel=7284631
\bibitem{[2]}
\textit{How does a door/window alarm sensor actually work.} \\www.alarmgrid.com/faq/how-does-a-door-slash-window-alarm-sensor-actually-work
\end{thebibliography}


\end{document}
