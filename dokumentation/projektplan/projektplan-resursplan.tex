\documentclass[a4paper]{article}

\usepackage[swedish]{babel}
\usepackage[T1]{fontenc}
\usepackage[utf8]{inputenc}

\begin{document}

\section*{RESURSPLAN}
Projektets ansvarområden är uppdelat till medlemmarna enligt lista nedan: 
\begin{itemize}
\item Christoffer Kaltenbrunner - chrkalt@student.chalmers.se: 
	\begin{itemize}
	\item Gruppledare: Ansvarar för kommunikationen inom gruppen och mellan gruppen och externa kontakter. Kallar till och leder gruppmöten.
	\item Testansvarig: Ansvarar för olika tester som behöver genomföras på mjuk- och hårdvara, ser till att tillräcklig tid ägnas åt test och att testerna är väldokumenterade. 
	\end{itemize}
\item Daniel Ferreira - danfer@student.chalmers.se:
	\begin{itemize}
	\item Tekniskt dokumentationsansvarig: Ser till att kod är väl kommenterad och uppmärksammar gruppen ifall det uppstår problem.
	\item Kodstandardansvarig: Ansvarar för kodutveckling, ser till att koden förljer standarder för struktur, namn och så vidare för att uppnå kodkvalitet.
	\end{itemize}
\item Thanh Lam Nguyen - thanhl@student.chalmers.se: 
	\begin{itemize}
	\item Resursansvarig: Ser till att hårdvara och olika verktyg som används för kommunikation, versionshantering och så vidare finns tillgängliga och fungerar väl.
	\end{itemize}
\item Erik Söderpalm - eriks@student.chalmers.se:
	\begin{itemize}
	\item Administrativt dokumentationsansvarig: Ansvarar för att mötesprotokoll, oppositionsrapport, planeringsrapport och så vidare förs och skickas in i tid. 
	\end{itemize}
\item Jakob Wik - wikj@student.chalmeres.se:
	\begin{itemize}
	\item Planeringsansvarig: Håller reda på de olika delmål gruppen arbetar mot, kommunicerar med de andra medlemmarna om hur arbetet går och uppmärksammar vid behov gruppen ifall det inte görs framsteg mot något mål, så att gruppen kan omfördela arbetsinsatser.
	\end{itemize}
\end{itemize}

Alla veckomöten och testtillfällen kommer att hållas i rum EG-4213 i EDIT-huset i mån av lokaltillgänglighet, med reservation för ändring. Grupprum som kan användas vid exempelvis gruppkodning, småmöten och så vidare, finns tillgängliga för bokning i EDIT-huset vid behov. \\

Mottagen hårdvara till projektet: 
\begin{itemize}
	\item 3x MD407 kort
	\item 1x Ultraljud avståndsmätare, HC-SR04
	\item 1x Vibrationssensor, "Flying Fish" SW-18010P
	\item 1x Keypad
	\item 1x 7-segmentsdisplay
	\item 2x 4-polig RJ-11 kabel (används för CAN-bussen)
	\item 1x RJ-11 förgrening
	\item 2x Tiopolig flatkabel
	\item 3x USB-kabel
	\item 1x Kopplingsplatta
	\item Kopplingskablar (finns i projektrummen).
\end{itemize}

Hårdvaran är inlåst i ett säkerhetsskåp i rum EG-4213. Skåpet låses upp med PIN-kod som alla gruppmedlemmar har tillgång till. Utöver ovan nämnd hårdvara finns det möjlighet att låna annan utrustning, till exempel dörrsensor.
GitHub används som versionshanteringssystem för projektet. Ett privat fjärrepo på GitHub har skapats och delats med hela gruppen. All projektrelaterad kod och dokumentation kommer att finnas där.

\end{document}
