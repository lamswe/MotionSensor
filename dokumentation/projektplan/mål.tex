\documentclass[a4paper]{article}

\usepackage[swedish]{babel}
\usepackage[T1]{fontenc}
\usepackage[utf 8]{inputenc}

\begin{document}

\section*{Mål}

Målet med projektet är att utveckla ett larm- och låssystem med hjälp av en mikrokontroller och ett antal periferienheter.
Larm- och låssystemet består av ett grundsystem med ett antal tilläggsfunktioner.

Grundsystemet utgörs av ett dörrlarm, ett rörelselarm samt en störenhet för teständamål. Inställningarna för larmenheter kan konfigureras via en centralenhet.

Tilläggsfunktionerna delas in i två delar: (1) viktiga funktioner och (2) mindre viktiga funktioner. De viktiga funktionerna omfattar en funktion att slå om från produktionsläge till testläge för enklare test av periferienheterna samt en funktion som gör systemet självlärande.  De mindre viktiga funktionerna omfattar RSA-kryptering av meddelanden som skickas i systemet samt en uppseendeveckande larmsignal.
I första hand ska de viktiga funktionerna utvecklas.

\end{document}

