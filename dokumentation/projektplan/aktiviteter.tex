\documentclass[a4paper]{article}

\usepackage[swedish]{babel}
\usepackage[T1]{fontenc}
\usepackage[utf8]{inputenc}

\begin{document}

\section{Aktiviteter}
I tabell (nr) ges en aktivitetslista med ungefärlig tidsåtgång som varje aktivitet kräver.

Läsa igenom dokumentation för md-407 och resterande hårdvara uppskattas till att inte ta mer än en halv arbetsdag.

Gruppen har för avsikt att träffas över ett projektmöte i veckan. På dessa möten ska projektet ses över och vad alla har producerat/ändrat senaste veckan.

De första två veckorna förmodas nästan all tid gå åt att skriva projektplanen, detta innebär att alla lägger cirka 40 timmar var.

Planen är att vara klara med bassystemet inom två veckor från den första kodutvecklingsdagen. Under dessa två veckor ska samtidigt systemet testas och tillhörande testspecifikationer skrivas. Övrig tid under dessa veckor ägnas åt rapportskrivningen.

När bassystemet är färdigutvecklat har gruppen för avsikt att utveckla \\ytterligare funktioner över den nästföljande veckan samt utföra tester på dessa.

Dokumentation och kodförbättring förmodas ta cirka en arbetsvecka.

Andra utkastet av rapporten uppskattas till att ta cirka halva tiden som gick åt att skriva det första, då det förhoppningsvis endast är några få korrigeringar som behöver göras.

Demoförberedelserna estimeras till att ta cirka två dagars arbetstid att \\genomföra. Demon kommer i sig ta cirka två timmar och alla i projektgruppen är närvarande.

Slutligen är gruppens mål att slutrapporten bara kräver några enkla\\ korrigeringar, som beräknas ta cirka två dagars arbetstid.

Gruppen har för avsikt att på de veckoliga mötena dela ut arbetsuppgifter utifrån den rådande situationen. Alla medlemmar är eniga om att detta system är fördelaktigare, eftersom det är svårt att bestämma vilka aktiviteter som ska utföras av vem redan nu. Aktivitetslistan är därmed översiktlig.
\\
\\
\begin{table}[t]
\begin{tabular}{| l | l | l |}
\hline
Nr & Beskrivning & Tidsåtgång \\ \hline
1 & Läsa dokumentation & 20 h \\ \hline
2 & Projektmöten (1-1.5h/vecka, 8 veckor, 5 personer) & 50 h \\ \hline
3 & Skapa GitHub-repo & 1 h \\ \hline
4 & Skriva projektplan & 200 h \\ \hline
5 & Utveckling av bassystem & 100 h \\ \hline
6 & Rapportutkast 1 & 100 h \\ \hline
7 & Testspecifikationer & 10 h \\ \hline
8 & Tester av bassystem & 50 h \\ \hline
9 & Utveckling av ytterligare funktioner & 100 h \\ \hline
10 & Tester av det utvecklade systemet & 50 h \\ \hline
11 & Oppositionskommentar & 10 h \\ \hline
12 & Dokumentation och kodförbättring & 100 h \\ \hline
13 & Rapportutkast 2 & 50 h \\ \hline
14 & Demoförberedelser & 50 h \\ \hline
15 & Demo (2h/person, 5 personer) & 10 h \\ \hline
16 & Slutrapport & 50 h \\ \hline
\end{tabular}
\caption{Aktivitetslista och den uppskattade tidsåtgången för varje aktivitet i totala mantimmar}
\end{table}



\end{document}
